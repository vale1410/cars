\documentclass[]{llncs} 
\usepackage{amsmath}
\usepackage{verbatim}
\usepackage{datatool}
\usepackage{tikz}
\usetikzlibrary{arrows,shapes}
\usepackage{graphics}


\newcommand{\TODO}[1]{ {\color{red}{#1} }}
\newcommand{\com}[1]{ {\color{blue}{--- #1 ---}}}

\newcommand{\AtMostSeqCard}{AtMostSeqCard }
%
%\author{Valentin Mayer-Eichberger}
%
%\institute{NICTA \\ University of New South Wales \\
%\email{valentin.mayer-eichberger@nicta.com.au}}

\title{Modelling in Propositional Logic: A Case Study}

\begin{document} 

\maketitle

\begin{abstract}
    Car sequencing has been a traditional benchmark in the Operation Research and Constraint Programming community. In
    this paper we will demonstrate that state of the art propositional satisfiability solvers can compete with these
    established paradigms.  We will show how to express the traditional car sequencing problem in different ways in
    propositional logic and give an empirical evaluation of these translations. The results underline the practical
    challenge how NP-complete problems are reduced to SAT. 
\end{abstract}

\section{Introduction}

Contributions of the paper: 

1) to the (best of our konwledge) first SAT model for car sequencing and 2) demonstration of its usefulness on experiments
with the CSP lib 3) a comprehensive future work secion with future work and promising research directives. 

\section{SAT Solving}

Formally state the SAT problems. Give reference to NP compleness problems and reductions. Give reference to Handbook. 
Short overview of techniques of current state of the art SAT solvers. 

\subsection{Modelling in SAT}

Discribe the challenges in boolean modelling. Declarativity of the problem. Propagators of CP are rather procedural
descriptions of the reasoning task. Low level view and give advantages when tweaking the model, which can make the
difference for hard instances. 

Give list of techniques on what to focus on with encodings. Modelling in SAT is not just translating to CNF and a solver
will find the solution. 1) Design of variables 2) identifying higher constraints 3) espressing these constraints by
introducing auxiliary variables and 4) introduce redundant constraints and break symmetries. 

\subsection{Car Sequencing}

Describe the problem in short words and give references and their approaches to solve the probelems. Then reference the
benchmark set in \cite{Gent99}. 

\subsection{Literature Review}

Numerous publications on approaches to CS. 

Tradintional Models for CP. Give short formula. 

Give CP and MIP formulations. reference latest good results on the benchmark by a more global view on the problems. 

\subsection{Counter Encoding}

First we will show how to do a counter encoding. 

\section{Car Sequencing in CNF}

\section{Evaluation}

\section{Conclusion and  Future Work}


\bibliography{p}
\bibliographystyle{apalike}

\end{document}
