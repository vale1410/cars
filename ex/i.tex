\documentclass[]{llncs} 
\usepackage{amsmath}
\usepackage{verbatim}
\usepackage{datatool}
\usepackage{tikz}
\usetikzlibrary{arrows,shapes}
\usepackage{graphics}


\newcommand{\TODO}[1]{ {\color{red}{#1} }}
\newcommand{\com}[1]{ {\color{blue}{--- #1 ---}}}

\begin{document}

\section{Preprocessing Lower Bounds}

The Idea to this method goes back to the proof in \cite{Gent98} to show a lower bound of 2 for the instance 19/97. Here
we will show how to generalize this technique and apply this method on all problems from the benchmark in \cite{Gravel05}. 

The idea is to compute the demand for subsets of options and its adjacent classes. If such a class fulfills certain
properties, then we can compute the demand of cars that have none of the options. 


The following three cases lead to lower bounds: 

\begin{itemize}
    \item A set of options $O$ that contains only $1/q$ capacity constraints and at least on of them is $1/2$. Let the
        collective demand for this set of options be $k$, then for a legal sequence of cars there have to be at least
        $k-1$ cars that do not have any of the options in $O$. 
    \item A set of options $O$ that contains only $1/q$ with $q \geq 3$ capacity constraints. Let the collective demand
        for this set of options be $k$, then for a legal sequence of cars there have to be at least $2\cdot (k-1_)$ cars
        that do not have any of the options in $O$. 
    \item A set of options $O$ that contain at least one $1/q$ where $q \geq 3$ and exactly one $2/r$ where $r \geq 3$
        and arbitrary many $1/s$ constraints. If the demand for this set is $k$, then there have to be at least $k-1$ cars
        that have none of these options in $O$. 
\end{itemize}

\begin{proposition}
    For all three cases above, the difference to the demand and availability of adjacent cars give a lower bound on
    the violation of capacity constraints.
\end{proposition}


\section{New Lower Bounds}

The following table shows the known upper bounds and lower bounds published in the following works \cite{Regin97},
\cite{Gent98}. \cite{Gottlieb03}, \cite{Gravel05}, \cite{Estellon06}.

\begin{table}[htbp]
    \caption{}
    \centering
    %%%%%%%%%%%%%%%%%%%%%%%%%%%%%%%%%%%%%%%%%%%%%%%%%%%%%%%%%%%%%%%%%%%%%%
%%                                                                  %%
%%  This is a LaTeX2e table fragment exported from Gnumeric.        %%
%%                                                                  %%
%%%%%%%%%%%%%%%%%%%%%%%%%%%%%%%%%%%%%%%%%%%%%%%%%%%%%%%%%%%%%%%%%%%%%%
\begin{tabular}{ | l || r | r || r | r | }
  \hline                        
instance &(prepro) lb	&minisat	&known lb	& known up\\
  \hline                        
4/72	&	&	&0	&0\\
6/76	&	&	&1	&6\\
10/93	&	&	&1	&3\\
16/81	&	&	&0	&0\\
19/71	&	&	&2	&2\\
21/90	&2	&	&1	&2\\
36/92	&	&	&1	&2\\
41/66	&	&	&0	&0\\
26/82	&	&	&0	&0\\
  \hline                        
200\_01	&	&	&	&0\\
200\_02	&2	&1	&	&2\\
200\_03	&	&1	&	&3\\
200\_04	&7	&1	&	&7\\
200\_05	&	&	&	&6\\
200\_06	&6	&	&	&6\\
200\_07	&	&	&	&0\\
200\_08	&8	&1	&	&8\\
200\_09	&10	&1	&	&10\\
200\_10	&17	&	&	&19\\
300\_01	&	&	&	&0\\
300\_02	&	&	&	&12\\
300\_03	&13	&1	&	&13\\
300\_04	&7	&1	&	&7\\
300\_05	&2	&1	&	&28\\
300\_06	&2	&	&	&2\\
300\_07	&	&	&	&0\\
300\_08	&8	&1	&	&8\\
300\_09	&7	&	&	&7\\
300\_10	&3	&1	&	&21\\
400\_01	&	&	&	&1\\
400\_02	&15	&	&	&15\\
400\_03	&	&1	&	&9\\
400\_04	&19	&1	&	&19\\
400\_05	&	&	&	&0\\
400\_06	&	&	&	&0\\
400\_07	&	&	&	&4\\
400\_08	&4	&	&	&4\\
400\_09	&	&1	&	&5\\
400\_10	&	&	&	&0\\
  \hline                        
\end{tabular}



    \label{tab:1}
\end{table}

\bibliography{p}
\bibliographystyle{apalike}

\end{document}
