\documentclass[]{llncs} 
\usepackage{amsmath}
\usepackage{verbatim}
\usepackage{datatool}
\usepackage{tikz}
\usetikzlibrary{arrows,shapes}
\usepackage{graphics}


\newcommand{\TODO}[1]{ {\color{red}{#1} }}
\newcommand{\com}[1]{ {\color{blue}{--- #1 ---}}}

\begin{document}

\section{Preprocessing Lower Bounds}

The Idea to this method goes back to the proof in \cite{Gent98} to show a lower bound of 2 for the instance 19/97. Here
we will show how to generalize this technique and apply this method on all problems from the benchmark in \cite{Gravel05}. 

The idea is to compute the demand for all classes that share a subset of options and its potential adjacent classes. If
such a class fulfills certain properties, then we can compute the demand of cars that have none of the options. If this
demand is not met, then we have found a lower bound. 

The following three cases can lead to lower bounds: 

\begin{itemize}
    \item A subset of options $B\subseteq O$ that contains only  capacity constraints with $1/q$ and at least one with
        $1/2$. Let the collective demand for this set of options be $k$, then for a legal sequence of cars there have to
        be at least $k-1$ cars that do not have any of the options in $B$. 
    \item A subset of options $B\subseteq O$ that contains only capacity constraints with $1/q$ with $q \geq 3$. Let the
        collective demand for this set of options be $k$, then for a legal sequence of cars there have to be at least
        $2\cdot (k-1)$ cars that do not have any of the options in $B$. 
    \item A set of options $B\subseteq O$ that contain at least one capacity constraint with $1/q$ where $q \geq 3$ and
        exactly one with $2/r$ where $r \geq 3$ and arbitrary many $1/s$ constraints. If the demand for this set is $k$,
        then there have to be at least $k-2$ cars that have none of these options in $O$. 
\end{itemize}

\begin{proposition}
    For all three cases above, the difference to the demand and availability of adjacent cars give a lower bound on
    the violation of capacity constraints.
\end{proposition}

TODO: proof and example

\section{New Lower Bounds}

The following table shows the known upper bounds and lower bounds published in the following works \cite{Regin97},
\cite{Gent98}. \cite{Gottlieb03}, \cite{Gravel05}, \cite{Estellon06}, as well as results from running minisat with dummy
cars. The number of dummy cars ranges from 0 to the best known upper bound. Each run was limited by 1800 seconds. Take
into account that the SAT approach computes bounds by adding dummy cars and such the lower bounds are also lower bounds
for the other definitions of the optimization goal. Upper bounds cannot be compared, as adding dummy cars is less strict
than minimizing the violated capacity constraints.

\begin{table}[htbp]
    \caption{Lower and upper bounds found by the Ian Gent trick lb (preproc), the SAT encoding with minisat and the best known.}
    \centering
    %%%%%%%%%%%%%%%%%%%%%%%%%%%%%%%%%%%%%%%%%%%%%%%%%%%%%%%%%%%%%%%%%%%%%%
%%                                                                  %%
%%  This is a LaTeX2e table fragment exported from Gnumeric.        %%
%%                                                                  %%
%%%%%%%%%%%%%%%%%%%%%%%%%%%%%%%%%%%%%%%%%%%%%%%%%%%%%%%%%%%%%%%%%%%%%%
\begin{tabular}{ l|c|cc|cc|cc  }
Instance &LB (pre)	&LB (SAT)	&Time &UB (SAT)	&Time & LB*(known)   & UB*(known) \\
    \hline
4/72	&	&0	&	&0	&0.10	&0	&0	\\
6/76	&	&6	&673.87	&6	&0.05	&1	&6	\\
10/93	&	&1	&7.70	&3	&0.18	&1	&3	\\
16/81	&	&0	&	&0	&0.07	&0	&0	\\
19/71	&2	&	& timeout &2	&0.39	&2	&2	\\
21/90	&2	&1	&150.66	&2	&0.11	&1	&2	\\
36/92	&	&1	&31.61	&1	&0.21	&1	&2	\\
41/66	&	&0	&	&0	&0.04	&0	&0	\\
26/82	&	&0	&	&0	&0.10	&0	&0	\\
200\_01	&	&0	&	&0	&33.11	&	&0	\\
200\_02	&2	&1	&33.11  &2	&1.39	&	&2	\\
200\_03	&	&2	&1258.10	&3	&28.33	&	&3	\\
200\_04	&7	&1	&4.36	&7	&3.92	&	&7	\\
200\_05	&	&1	&1109.32	&3	&2.58	&	&6	\\
200\_06	&6	&	& timeout  &6	&9.55	&	&6	\\
200\_07	&	&0	&	&0	&0.62	&	&0	\\
200\_08	&8	&	& timeout  &8	&5.46	&	&8	\\
200\_09	&10	&1	&496.46	&10	&1.56	&	&10	\\
200\_10	&17	&14	&798.10	&17	&6.28	&	&19	\\
300\_01	&	&0	&	&0	&3.74	&	&0	\\
300\_02	&	&	&	&6	&17.60	&	&12	\\
300\_03	&13	&2	&586.03	&13	&15.03	&	&13	\\
300\_04	&7	&6	&274.48	&7	&21.48	&	&7	\\
300\_05	&2	&10	&1101.27	&17	&899.48	&	&27	\\
300\_06	&2	&	& timeout  &2	&337.12	&	&2	\\
300\_07	&	&0	&	&0	&1.59	&	&0	\\
300\_08	&8	&1	&102.26	&8	&30.37	&	&8	\\
300\_09	&7	&	& timeout &7	&5.81	&	&7	\\
300\_10	&3	&8	&1200.46	&12	&1426.53	&	&21	\\
400\_01	&	&	& timeout   &	& timeout  &	&1	\\
400\_02	&15	&	& timeout  &15	&62.69	&	&15	\\
400\_03	&	&8	&1543.67	&	& timeout  &	&9	\\
400\_04	&19	&5	&60.13	&19	&550.03	&	&19	\\
400\_05	&	&0	&	&0	&143.58	&	&0	\\
400\_06	&	&0	&	&0	&20.96	&	&0	\\
400\_07	&	&	& timeout  &	& timeout  &	&4	\\
400\_08	&4	&	& timeout  &	& timeout  &	&4	\\
400\_09	&	&4	&633.61	&5	&57.80	&	&5	\\
400\_10	&	&0	&	&0	&4.28	&	&0	\\
\hline                        
\end{tabular}

    \label{tab:1}
\end{table}

\bibliography{p}
\bibliographystyle{apalike}

\end{document}
