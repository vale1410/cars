\documentclass[conference]{IEEEtran}
\usepackage{amsmath}
\usepackage{verbatim}
\usepackage{datatool}
\usepackage{tikz}
\usetikzlibrary{arrows,shapes}
\usepackage{graphics}

\newcounter{Examplecount}
\setcounter{Examplecount}{0}
\newenvironment{example}
{% This is the begin code
    \stepcounter{Examplecount} Example \arabic{Examplecount} 
\begin{it}
    } 
    {% This is the end code
\end{it} 
}

\begin{document}
%
% paper title
% can use linebreaks \\ within to get better formatting as desired
\title{SAT Benchmark for the Car Sequencing Problem}


% author names and affiliations
% use a multiple column layout for up to three different
% affiliations
\author{\IEEEauthorblockN{Valentin Mayer-Eichberger}
\IEEEauthorblockA{NICTA and\\University of New South Wales\\
Valentin.Mayer-Eichberger@nicta.com.au}}

% make the title area
\maketitle


\begin{abstract}
    Car sequencing occurs in the production process of the automotive industry. It addresses the problem of scheduling
    cars along an assembly line such that capacities of different workstations along the line are not exceeded. We
    provide the SAT competition with a selection of hard car sequencing problems from the CSP$_{\mbox{LIB}}$
    \cite{Gent99}. The encoding is based on a variant of the sequential counter encoding of cardinality constraints and
    the reuse of auxiliary variables.  
\end{abstract}
\IEEEpeerreviewmaketitle

\section{Introduction}

Car sequencing deals with the problem of scheduling cars along an assembly line with capacity constraints for different
stations (e.g. radio, sun roof, air-conditioning, etc). Cars are partitioned into classes according to their
requirements. The stations are denoted as \emph{options} and defined by a ratio $u/q$ restricting the maximal number $u$
of cars that can be scheduled on every subsequence of length $q$.

\begin{example} 
    Given classes $C = \{1,2,3\}$ and options $O = \{a,b\}$. The demands (number of cars) for the classes are $3,2,2$,
    respectively. Capacity constraints on options are given by $a:1/2$ and $b:1/5$, respectively. Class 1 has no restrictions,
    class 2 requires option $a$ and class 3 needs options $\{a, b\}$. The only legal sequence for this problem is
    $[3,1,2,1,2,1,3]$, since class 2 and 3 cannot be sequenced after another and class 3 need to be at least 5 positions
    apart.
\end{example}                                     

Car sequencing in the CSPlib contains a selection of benchmark problems of this form ranging from 100 to 400 cars. Over
the years different approaches have been used to solve these instances, among them constraint programming, local search
and integer programming \cite{Regin97}\cite{Gottlieb03}\cite{Gravel05}\cite{Estellon06}\cite{Siala12}.

Car sequencing has also been treated as an optimisation problem and several versions for the opimisation goal have been
proposed. Most of the approaches use a variant of minimising the number of violated capacity constraints.  However, for
this benchmark we use the definition of \cite{Perron04} which transforms easily to sequence of decision problems and SAT
solving can be directly applied: An unsatisfiable car sequencing problem can be made solvable by adding empty slots to
the sequence. The goal is then to minimise the number of empty slots needed for a valid sequence. A lower bound $lb$ is
proven by unsatisfiability with $lb-1$ additional empty slots. 

\section{The Encoding}

The car sequencing problem can be naturally modelled by Boolean cardinality constraints. Our approach is to  translate
cardinality constraints by a variant of the sequential counter encoding proposed by \cite{Sinz05}. The key idea is then
to integrate capacity constraint into the sequential counter of the demand constraints by reusing the auxiliary
variables. This enforces a global view on the conjunction of these two constraints and facilitates propagation. Our own
experiments show that this encodings is far better than naive approaches or an automatic translation from the pseudo
Boolean model. 

\section{The Benchmark}

A command line tool that generates CNF in DIMACS format from a problem description in the CSPlib is freely available at
\verb+github.com/vale1410/car-sequencing+. With this tool one can generate different encodings and compare the runtime
of SAT solvers. For this benchmark we chose the best encoding according to our experiments and we are interested if the
solvers from the competition are able to prove stronger bounds. 

\section*{Acknowledgement}

NICTA is funded by the Australian Government as represented by the Department of Broadband, Communications and the
Digital Economy and the Australian Research Council through the ICT Centre of Excellence program.


\bibliographystyle{IEEEtran}
\bibliography{p}

\end{document}
