\documentclass[conference]{IEEEtran}

\begin{document}
	
% paper title
\title{MySolver}

% author names and affiliations
% use a multiple column layout for up to three different
% affiliations
\author{\IEEEauthorblockN{Michael Shell}
\IEEEauthorblockA{Georgia Institute of Technology\\
  Atlanta, Georgia}
\and
\IEEEauthorblockN{Homer Simpson}
\IEEEauthorblockA{Twentieth Century Fox\\
  Springfield, USA}
\and
\IEEEauthorblockN{James Kirk\\ and Montgomery Scott}
\IEEEauthorblockA{Starfleet Academy\\
  San Francisco, CA, USA}
}

% conference papers do not typically use \thanks and this command
% is locked out in conference mode. If really needed, such as for
% the acknowledgment of grants, issue a \IEEEoverridecommandlockouts
% after \documentclass

% for over three affiliations, or if they all won't fit within the width
% of the page, use this alternative format:
% 
%\author{\IEEEauthorblockN{Michael Shell\IEEEauthorrefmark{1},
%Homer Simpson\IEEEauthorrefmark{2},
%James Kirk\IEEEauthorrefmark{3}, 
%Montgomery Scott\IEEEauthorrefmark{3} and
%Eldon Tyrell\IEEEauthorrefmark{4}}
%\IEEEauthorblockA{\IEEEauthorrefmark{1}School of Electrical and Computer Engineering\\
%Georgia Institute of Technology,
%Atlanta, Georgia 30332--0250\\ Email: see http://www.michaelshell.org/contact.html}
%\IEEEauthorblockA{\IEEEauthorrefmark{2}Twentieth Century Fox, Springfield, USA\\
%Email: homer@thesimpsons.com}
%\IEEEauthorblockA{\IEEEauthorrefmark{3}Starfleet Academy, San Francisco, California 96678-2391\\
%Telephone: (800) 555--1212, Fax: (888) 555--1212}
%\IEEEauthorblockA{\IEEEauthorrefmark{4}Tyrell Inc., 123 Replicant Street, Los Angeles, California 90210--4321}}

% use for special paper notices
%\IEEEspecialpapernotice{(Invited Paper)}

\maketitle

% the abstract is optional
\begin{abstract}
This document describes the SAT solver ``MySolver'', a new kind of hybrid solver combining
local search, CDCL, and survey propagation.
\end{abstract}

\section{Introduction}

This instructions provide general guidelines on what a good solver description contains. 
The sectioning may be changes, as long as the required details are presented.

Notice that the solver description should be specific to the particular version of your solver that 
is submitted to SAT Competition 2013. Even if you have previously published a paper
on a previous version of your solver, simply providing such an earlier paper 
or just referring to such a paper does not meet the requirements for solver descriptions
for SAT Competition 2013.

Notice also that, following the principles of scientific writing, necessary references
to known techniques implemented in your solver should be provided.
Example reference: \cite{JeroslowWang:1990}

Naming convention for the solver description PDF: 
name the file according to the name of your solver.

Make sure that page numbering is turned off.

\section{Main Techniques}

Which algorithmic paradigm(s) the solver is based on: CDCL, SLS, look-ahead, hybrid (of what), portfolio (of what type of solvers, which solvers) ? 

What further solving techniques are used (e.g. preprocessing, restart/learning/... strategies, ...)?


\section{Main Parameters}



\begin{enumerate}
\item What are the performance-sensitive parameters (both under user control and 
internally-used) and what do they control? 
\item Are there any ``magic constants''? What are they?
\item What values do these parameters take for the competition? 
\item Are the parameters dependent on instance properties? If yes, provide 
on the properties and how they are used.
\item \ldots
\end{enumerate}

Pay special attention to the parameters you have tuned (by hand or automatically) 
for SAT Competition 2013.

\section{Special Algorithms, Data Structures, and Other Features}

Implementation-level details: special data structures, algorithmic details, \ldots


\section{Implementation details}

\begin{enumerate}
\item In which program language(s) is the solver implemented in?
\item Was the solver implemented from scratch, or is it based on other solver(s)? Which solver?
\item \ldots
\end{enumerate}
 
\section{SAT Competition 2013 Specifics}
\begin{enumerate}
  \item In which tracks was the solver submitted to?
  \item Which compiler (including version) was used?
  \item What optimization flags were used in compilation?
  \item 32-bit or 64-bit binary?
  \item Command-line options? Which solver parameters were set to which values?
\item \ldots
\end{enumerate}

\section{Availability}

\begin{enumerate}
  \item Is the solver open source, publicly available? Under which license?
  \item Provide a URL from which the solver can be downloaded
\item \ldots
\end{enumerate}




\section*{Acknowledgment}
The authors would like to thank...

\bigskip
What should not be in the system description:
\begin{enumerate}
  \item Basic definitions related to SAT. (However, any formal notations used in the description should be defined.)
  \item Empirical results on the solver's performance.
\end{enumerate}

\bibliographystyle{IEEEtran}
\bibliography{sc2013}

\end{document}


