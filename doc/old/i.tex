\documentclass[]{llncs} 
\usepackage{amsmath}
\usepackage{verbatim}
\usepackage{datatool}
\usepackage{tikz}
\usetikzlibrary{arrows,shapes}
\usepackage{graphics}


\newcommand{\TODO}[1]{ {\color{red}{#1} }}
\newcommand{\com}[1]{ {\color{blue}{--- #1 ---}}}

\begin{document}

\section{Preprocessing Lower Bounds}

The idea to this method goes back to the proof in \cite{Gent98} to show a lower bound of 2 for the instance 19/97. Here
we will show how to generalize this technique and apply this method on all problems from the benchmark in \cite{Gravel05}. 

We start with instance 300-04 as an example. The demands and options are given in Table \ref{tab:2}. 

\begin{table}[htbp]
    \caption{Overview of options and demands for instance 300-04}
    \centering
    %%%%%%%%%%%%%%%%%%%%%%%%%%%%%%%%%%%%%%%%%%%%%%%%%%%%%%%%%%%%%%%%%%%%%%
%%                                                                  %%
%%  This is a LaTeX2e table fragment exported from Gnumeric.        %%
%%                                                                  %%
%%%%%%%%%%%%%%%%%%%%%%%%%%%%%%%%%%%%%%%%%%%%%%%%%%%%%%%%%%%%%%%%%%%%%%
\begin{tabular*}{1\linewidth}{@{\extracolsep{\fill}} c|rrrrrrrrrrrr}
class	&0	&1	&2	&3	&4	&5	&6	&7	&8	&9	&10	&11	\\
demand	&9	&4	&22	&2	&1	&62	&31	&4	&24	&4	&3	&36	\\
  \hline                        
0: $1/2$	&-	&-	&-	&-	&-	&-	&-	&-	&-	&-	&-  &x \\
1: $2/3$	&-	&-	&-	&-	&-	&x	&x	&x	&x	&x	&x  &- \\
2: $1/3$	&-	&-	&x	&x	&x	&-	&-	&-	&x	&x	&x  &- \\
3: $2/5$	&-	&x	&-	&-	&x	&-	&x	&x	&-	&-	&x  &- \\
4: $1/5$	&x	&x	&-	&x	&x	&-	&-	&x	&-	&x	&x  &- \\
\end{tabular*}

\vspace{1cm}

\begin{tabular*}{1\linewidth}{@{\extracolsep{\fill}} c|rrrrrrrrrrrrr}
class		&12	&13	&14	&15	&16	&17	&18	&19	&20	&\textbf{21} &22 &\textbf{23}\\
demand		&3	&25	&3	&8	&5	&2	&6	&21	&5	&\textbf{7 } &11 &\textbf{2}\\
  \hline
0: $1/2$	&x	&x	&x	&x	&x	&x	&x	&x	&x	&\textbf{x } &x	 &\textbf{x}\\
1: $2/3$	&-	&-	&-	&-	&-	&-	&x	&x	&x	&\textbf{x } &x	 &\textbf{x}\\
2: $1/3$	&-	&-	&-	&x	&x	&x	&-	&-	&-	&\textbf{x } &x	 &\textbf{x}\\
3: $2/5$	&-	&x	&x	&-	&-	&x	&-	&x	&x	&\textbf{- } &x	 &\textbf{x}\\
4: $1/5$	&x	&-	&x	&-	&x	&x	&x	&-	&x	&\textbf{x } &-	 &\textbf{x}\\
\end{tabular*}

    \label{tab:2}
\end{table}

There are two classes, 21 and 23, that require options 0, 1, 2 and 4 and sum of demands is 9. First observation is that
all other classes share at least one options with these two classes.  Secondly cars of class 21 and 23 have to be put at
least 5 apart, so they cannot share a neighbour.  Second they cannot be neighbour to any of the classes that have a
$1/q$ restriction. This leaves us with the classes that only share the option 1 and for each car at most one adjacent
car can have restriction $2/3$. Since the first and the last car in the sequence can have any neighbour with that
restriction, the number of cars that share no option is at least $9-2=7$. Since there are no such cars, the lower bound
for violations (dummy cars or violated capacity constraints) is 7. 

A similar argument can be made for classes 21, 22, 23 that share options 0, 1 and 2. Here the collective demand is 20
and the supply of cars that have neither of these options is $20 - 13 = 7$. This gives a lower bound of 5, which is weaker than the
first case. 

The general idea is to compute the demand for classes that share a subset of options such that there is a lower bound on
the number of cars that do not share any options with this subset.

\begin{proposition}
    The following cases can be used to preprocess lower bounds. 
\begin{enumerate}
    \item A subset of options $B\subseteq O$ that contains only  capacity constraints of the form $1/q$ and at least one
        with $1/2$. Let the collective demand for this set of options be $k$, then for a legal sequence of cars there
        have to be at least $k-1$ cars that do not have any of the options in $B$. 
    \item A subset of options $B\subseteq O$ that contains only capacity constraints of the form $1/q$ with $q \geq 3$.
        Let the collective demand for this set of options be $k$, then for a legal sequence of cars there have to be at
        least $2\cdot (k-1)$ cars that do not have any of the options in $B$. 
    \item A set of options $B\subseteq O$ that contain at least one capacity constraint with $1/q$ where $q \geq 3$ and
        exactly one with $2/r$ where $r \geq 3$ and arbitrary many $1/s$ constraints. If the demand for this set is $k$,
        then there have to be at least $k-2$ cars that have none of these options in $O$. 
\end{enumerate}

\end{proposition}

\begin{proof}
    still working on it...
%    For convenience we describe sequences of cars by words of the alphabet $\{a,b,c\}$, $a$ denotes a car that requires
%    at least one option in $B$, $b$ denotes a car that requires all options in $B$ and $c$ denotes a car that requires
%    no option in $B$. 
%    \begin{enumerate}
%        \item For each car having all options in $B$ there cannot be cars next to it that require at least one option in
%            $B$.  Neighbours can be shared, so the sequence as $bcbcbc\ldots cb$ is one with least number of $c$s
%            necessary. Thus there have to be at least $k-1$ cars with no option of $B$. 
%        \item Similar to 1) but here neighbours cannot be shared. 
%        \item Neighbours cannot be shared and due to the capacity constraint with $2/q$ one of the neighbors. 
%    \end{enumerate}
\end{proof}


\section{New Lower Bounds}

The following table shows the known lower bounds (LB) and upper bounds (UB) published in the following works \cite{Regin97},
\cite{Gent98}. \cite{Gottlieb03}, \cite{Gravel05}, \cite{Estellon06}, as well as results from running minisat with dummy
cars. The number of dummy cars ranges from 0 to the best known upper bound. Each run was limited by 1800 seconds. Take
into account that the SAT approach computes bounds by adding dummy cars and such the lower bounds are also lower bounds
for the other definitions of the optimization goal. Upper bounds cannot be compared, as adding dummy cars is less strict
than minimizing the violated capacity constraints.

\begin{table}[htbp]
    \caption{Lower and upper bounds found by preprocessing (pre), by the SAT encoding and the best known. }
    \centering
    %%%%%%%%%%%%%%%%%%%%%%%%%%%%%%%%%%%%%%%%%%%%%%%%%%%%%%%%%%%%%%%%%%%%%%
%%                                                                  %%
%%  This is a LaTeX2e table fragment exported from Gnumeric.        %%
%%                                                                  %%
%%%%%%%%%%%%%%%%%%%%%%%%%%%%%%%%%%%%%%%%%%%%%%%%%%%%%%%%%%%%%%%%%%%%%%
\begin{tabular}{ l|c|cc|cc|cc  }
Instance &LB (pre)	&LB (SAT)	&Time &UB (SAT)	&Time & LB*(known)   & UB*(known) \\
    \hline
4/72	&	&0	&	&0	&0.10	&0	&0	\\
6/76	&	&6	&673.87	&6	&0.05	&1	&6	\\
10/93	&	&1	&7.70	&3	&0.18	&1	&3	\\
16/81	&	&0	&	&0	&0.07	&0	&0	\\
19/71	&2	&	& timeout &2	&0.39	&2	&2	\\
21/90	&2	&1	&150.66	&2	&0.11	&1	&2	\\
36/92	&	&1	&31.61	&1	&0.21	&1	&2	\\
41/66	&	&0	&	&0	&0.04	&0	&0	\\
26/82	&	&0	&	&0	&0.10	&0	&0	\\
200\_01	&	&0	&	&0	&33.11	&	&0	\\
200\_02	&2	&1	&33.11  &2	&1.39	&	&2	\\
200\_03	&	&2	&1258.10	&3	&28.33	&	&3	\\
200\_04	&7	&1	&4.36	&7	&3.92	&	&7	\\
200\_05	&	&1	&1109.32	&3	&2.58	&	&6	\\
200\_06	&6	&	& timeout  &6	&9.55	&	&6	\\
200\_07	&	&0	&	&0	&0.62	&	&0	\\
200\_08	&8	&	& timeout  &8	&5.46	&	&8	\\
200\_09	&10	&1	&496.46	&10	&1.56	&	&10	\\
200\_10	&17	&14	&798.10	&17	&6.28	&	&19	\\
300\_01	&	&0	&	&0	&3.74	&	&0	\\
300\_02	&	&	&	&6	&17.60	&	&12	\\
300\_03	&13	&2	&586.03	&13	&15.03	&	&13	\\
300\_04	&7	&6	&274.48	&7	&21.48	&	&7	\\
300\_05	&2	&10	&1101.27	&17	&899.48	&	&27	\\
300\_06	&2	&	& timeout  &2	&337.12	&	&2	\\
300\_07	&	&0	&	&0	&1.59	&	&0	\\
300\_08	&8	&1	&102.26	&8	&30.37	&	&8	\\
300\_09	&7	&	& timeout &7	&5.81	&	&7	\\
300\_10	&3	&8	&1200.46	&12	&1426.53	&	&21	\\
400\_01	&	&	& timeout   &	& timeout  &	&1	\\
400\_02	&15	&	& timeout  &15	&62.69	&	&15	\\
400\_03	&	&8	&1543.67	&	& timeout  &	&9	\\
400\_04	&19	&5	&60.13	&19	&550.03	&	&19	\\
400\_05	&	&0	&	&0	&143.58	&	&0	\\
400\_06	&	&0	&	&0	&20.96	&	&0	\\
400\_07	&	&	& timeout  &	& timeout  &	&4	\\
400\_08	&4	&	& timeout  &	& timeout  &	&4	\\
400\_09	&	&4	&633.61	&5	&57.80	&	&5	\\
400\_10	&	&0	&	&0	&4.28	&	&0	\\
\hline                        
\end{tabular}

    \label{tab:1}
\end{table}

\bibliography{p}
\bibliographystyle{apalike}

\end{document}
