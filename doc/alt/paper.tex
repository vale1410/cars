\documentclass{article}
\usepackage{amssymb}
\usepackage{amsmath}
\usepackage[utf8]{inputenc}
\usepackage{enumerate}

\title{Yet Another SAT Encoding for Car Sequencing} 
%\author{Valentin Mayer-Eichberger and Martin Gebser}

\usepackage{tikz}
\usetikzlibrary{arrows,shapes}
\usepackage{graphics}
\usepackage[showboxes]{textpos}

\def\qed{{\hfill $\Box$}}
\def\proof{{\noindent \textit{Proof.~}}}
\newtheorem{definition}{Definition}[section]
\newtheorem{lemma}{Lemma}[section]
\newtheorem{proposition}{Proposition}[section]
\newtheorem{theorem}{Theorem}[section]
\newtheorem{corollary}{Corollary}[section]
\newtheorem{example}{Example}[section]
\newtheorem{conjecture}{Conjecture}[section]

\def\constraint#1{\mbox{{\rm\sc #1}}}
\def\reachability{{\constraint{reachability }}}

%\def\edge{\operatorname{edge}}
%\def\reach{\operatorname{reach}}
%\def\path{\operatorname{path}}
%\def\start{\operatorname{start}}
%\def\stop{\operatorname{end}}
%\def\one#1{\constraint{Exactly1}(#1)}

\def\implies{\Rightarrow}
\def\when{\Leftarrow}
\def\and{\wedge}

\begin{document}

\maketitle

The encoding avoids a separate counter encoding for each cardinality
constraint and  at the same time is an alternative to \cite{Mayer13}. 

The encoding is wrt. one object (car or option) for which we omit the
subscript. We present an encoding for the constraint

$$\bigwedge_{i=0}^{n-q}(\sum_{l=1}^q x_{i+l} \leq u )$$

\begin{itemize}
    \item $x_i$ is true (1) if the object is at position $i$. 
    \item $s_{i,j}$ is true if in window $[i-q+1, \ldots i]$ there are at least $j$ objects. 
\end{itemize}

$$s_{i,j} \;\;\; \Leftrightarrow \;\;\; \sum_{l=i-q+1}^i x_l \geq j $$

The idea behind the clauses is to express the relationship between $x_{i-q}$,
$x_i$ and the counter $s_{i,j}$. 

\begin{align}
    \neg s_{i,j} & \vee s_{i+1,j-1} \\
    \neg s_{i+1,j+1} & \vee s_{i,j} \\
    \neg s_{i,j+1} & \vee s_{i,j} \\
    \neg x_i & \vee \neg s_{i-1,j} \vee s_{i,j} \\
         x_i & \vee \neg s_{i,j} \vee s_{i-1,j} \\
    \neg x_{i-q} & \vee \neg s_{i,j} \vee s_{i-1,j} \\
         x_{i-q} & \vee \neg s_{i-1,j} \vee s_{i,j} \\
    \neg x_i \vee x_{i-q} & \vee \neg s_{i-1,j-1} \vee s_{i,j} \\
         x_i \vee \neg x_{i-q} & \vee \neg s_{i,j} \vee s_{i-1,j+1}
\end{align}


\bibliographystyle{plain}
\bibliography{p}

\end{document}
